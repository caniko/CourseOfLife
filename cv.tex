\documentclass[10pt,a4paper,ragged2e,withhyper]{AltaCV/altacv}

\usepackage{hyperref}

% Change the page layout if you need to
\geometry{left=1.25cm,right=1.25cm,top=1.5cm,bottom=1.5cm,columnsep=1.2cm}

% The paracol package lets you typeset columns of text in parallel
\usepackage{paracol}

%% AltaCV uses the fontawesome and academicon fonts
%% and packages.
%% See texdoc.net/pkg/fontawecome and http://texdoc.net/pkg/academicons for full list of symbols. You MUST compile with XeLaTeX or LuaLaTeX if you want to use academicons.

% Change the font if you want to, depending on whether
% you're using pdflatex or xelatex/lualatex
\ifxetexorluatex
  % If using xelatex or lualatex:
  \setmainfont{Carlito}
\else
  % If using pdflatex:
  \usepackage[utf8]{inputenc}
  \usepackage[T1]{fontenc}
  \usepackage[default]{lato}
\fi
\usepackage{xcolor}

% Change the colours if you want to
\definecolor{RoyalBlue}{HTML}{2f688d}
\definecolor{SlateGrey}{HTML}{2E2E2E}
\definecolor{LightGrey}{HTML}{666666}
\colorlet{heading}{RoyalBlue}
\colorlet{accent}{RoyalBlue}
\colorlet{emphasis}{SlateGrey}
\colorlet{body}{LightGrey}

% Change the bullets for itemize and rating marker
% for \cvskill if you want to
\renewcommand{\itemmarker}{{\small\textbullet}}
\renewcommand{\ratingmarker}{\faCircle}

%% sample.bib contains your publications
% \addbibresource{sample.bib}


\begin{document}
\name{Can H. Tartanoglu}
\tagline{Agile Researcher \& Engineer | Ph. D. candidate in Medical Computation Neurosciences}
\photoR{2.8cm}{profile}

\personalinfo{
	% Not all of these are required!
	\location{Berlin, Germany}
	\email{canhtart@gmail.com}
	\phone{+49 176 64782175}
	%% \homepage{www.homepage.com}
	\twitter{@caniko}
	\linkedin{canhtartanoglu}
	\github{caniko}
	\orcid{0000-0001-5167-0678}
	%% You can add your own arbitrary detail with
	%% \printinfo{symbol}{detail}[optional hyperlink prefix]
	% \printinfo{\faPaw}{Hey ho!}[https://example.com/]
	%% Or you can declare your own field with
	%% \NewInfoFiled{fieldname}{symbol}[optional hyperlink prefix] and use it:
	% \NewInfoField{gitlab}{\faGitlab}[https://gitlab.com/]
	% \gitlab{your_id}
	%%
	%% For services and platforms like Mastodon where there isn't a
	%% straightforward relation between the user ID/nickname and the hyperlink,
	%% you can use \printinfo directly e.g.
	% \printinfo{\faMastodon}{@username@instace}[https://instance.url/@username]
	%% But if you absolutely want to create new dedicated info fields for
	%% such platforms, then use \NewInfoField* with a star:
	% \NewInfoField*{mastodon}{\faMastodon}
	%% then you can use \mastodon, with TWO arguments where the 2nd argument is
	%% the full hyperlink.
	% \mastodon{@username@instance}{https://instance.url/@username}
}

\makecvheader
%% Depending on your tastes, you may want to make fonts of itemize environments slightly smaller
% \AtBeginEnvironment{itemize}{\small}

%% Set the left/right column width ratio to 6:4.
\columnratio{0.6}

% Start a 2-column paracol. Both the left and right columns will automatically
% break across pages if things get too long.
\begin{paracol}{2}

\cvsection{Software experience}

	\cvevent{Software engineer and researcher}{On-Site at Haber Labaratory}{Aug 2020 -- Present}{Berlin, Germany}
		\begin{itemize}
			\item Leading the Synapse DB project. Infrastructure DevOps, full stack design and implementation, and executive decisions
			\item Assisted with automated animal behavior analysis in other Ph. D. projects
			\item Guided and assisted juniors and interns
		\end{itemize}

	\divider

	\cvevent{Open Source project author}{Remote}{May 2018 -- Present}{Oslo, Norway}
		FOSS projects in Python and Rust, explore my projects on \href{https://github.com/caniko}{GitHub}.

	\divider

	\cvevent{Software Engineer}{On-Site at Mnemonic}{May 2020 -- Aug 2020}{Oslo, Norway}
		One of two founders of the tooling team responsible for developing backend systems and interfaces for internal and B2B consumption.

	\divider

	\cvevent{Software Engineer}{Remote at Reshape Labs}{Feb 2019 -- Aug 2020}{}
		\begin{itemize}
			\item System design, company representative, and public speaker
			\item Created the first product of the company, which is a Tetris inspired game that is player VS player.
		\end{itemize}

	\divider

	\cvevent{Member of Group-EDB}{On-Site at the communication department in the Norwegian Student-community (DNS)}{March 2016 -- May 2017}{Oslo, Norway}
		Introduced to virtual machine infrastructure monitoring, and Linux ecosystem.

\cvsection{A Day of My Life}
	% Adapted from @Jake's answer from http://tex.stackexchange.com/a/82729/226
	% \wheelchart{outer radius}{inner radius}{
		% comma-separated list of value/text width/color/detail}
	% Some ad-hoc tweaking to adjust the labels so that they don't overlap
	\hspace*{-2em}  %% quick hack to move the wheelchart a bit left
	\wheelchart{1.5cm}{0.5cm}{
		10/13em/accent!30/Sleeping \& dreaming about disruption,
		25/9em/accent!60/Read write and correct scientific articles,
		5/11em/accent!10/\footnotesize\\[1ex]Guide and mentor,
		20/11em/accent!40/Spend time with my wife and family,
		5/8em/accent!20/Train for ultimate-marathon (73km),
		30/9em/accent/Engineer software and infrastructure,
		5/8em/accent!20/Play video games
	}

\newpage

\cvsection{Bioscience experience}

	\cvevent{Researcher for Academic projects in Neuroscience}{On-Site at the University of Oslo (UiO)}{March 2015 -- April 2020}{Oslo, Norway}
		\begin{itemize}
			\item The Prydz Laboratory: Worked with and maintained MDCK cells for glycobiology research.
			\item The Hafting Laboratory: Compute the time a mouse spends in the region of interest, created a new Python package DeepLabCutAnalysis.
			\item The Morland Laboratory (Institute of Pharmacology): Extraction of kinematic features from experimental video recordings of mice with induced stroke.
		\end{itemize}

	\divider

	\cvevent{Teaching Assistant in Bioscience courses at Bachelor level}{University of Oslo}{March 2016 -- Dec 2019}{Oslo, Norway}
		Bachelor courses including physiology, biochemistry, and computational modeling

	\divider

	\cvevent{Laboratory Technician for NoPSC in Melum Group}{Oslo University Hospital}{Sept 2016 -- Nov 2017}{Oslo, Norway}
		Worked with genotyping of mice, including isolation of DNA and amplification of target genes.

	\divider

	\cvevent{Intern at the Centre for Integrative Neuroplasticity}{University of Oslo}{Aug 2016 -- June 2017}{Oslo, Norway}
		Performed surgeries, partook in meetings, and preliminary work for Master's project

	\switchcolumn

	\cvsection{Life Philosophy}
		\begin{quote}
			``Embrace a holistic approach that integrates diverse perspectives and experiences``
		\end{quote}

	\cvsection{Achievements}
	
		\cvachievement{\faTrophy}{FOSS project founder, Oslo, Norway}{BiKiPy, a Python package, for analysing animal movement in behaviour experiments, four upcoming citations.}
		
		\cvachievement{\faTrophy}{Athletics}{Ran Toughest ORC, which is an 8 km run with challenging obstacles two times, 2016 and 2018. Oslo half-marathon in 2021.}

	\cvsection{Interests}

		\cvtag{Mentoring}
		\cvtag{Blockchain}
		\cvtag{Neuroscience}
		\cvtag{Engineering}
		\cvtag{Nutrition}
		\cvtag{Ultimate Frisbee}
		\cvtag{Windsurfing}
		\cvtag{Running}
		\cvtag{Fitness}
		
	\cvsection{Skills}

		\cvskill{Python}{5}
		\cvskill{Rust}{2.5}
		\cvskill{Latex}{3.5}
		\cvskill{FastAPI-OpenAPI}{4.5}
		\cvskill{Flutter/Dart}{4}

		\divider

		\cvskill{Arch Linux}{4}
		\cvskill{Kubernetes}{3.5}
		\cvskill{Cassandra DB}{3.5}
		\cvskill{PostgreSQL}{3}

	\cvsection{References}
	
		\cvref{Joakim von Brandis}{joakim@mnemonic.no}{Head of development, Mnemonic}

		\cvref{A/Prof. Torkel Hafting}{torkel.hafting@gmail.com}{Medical faculty, University of Oslo}

		\cvref{Prof. Espen Melum}{espen.melum@medisin.uio.no}{Medical faculty, University of Oslo}

	\cvsection{Education}
	
		\cvevent{Ph.D. Candidate}{German Center for Neurodegenerative Diseases (DZNE)}{Aug 2020 -- Present}{Berlin, Germany}
			Thesis title: "SynDB: Federated data platform for high-resolution microscopy data" \\
			Supervisors: Dr. Matthias Haberl, and Dr. Silvia Viana da Silva

		\divider

		\cvevent{Master of Science in Physiology}{University of Oslo}{July 2017 -- Dec 2019}{Oslo, Norway}
			Thesis title: "Markerless 3D tracking and reconstruction of subjects with multiple stereo cameras" \\
			Supervisors: A/Prof. Torkel Hafting, Dr. Alessio Buccino, Malin Benum Roe

		\divider
		
		\cvevent{Bachelor of Science in molecular biology and biochemistry}{University of Oslo}{July 2014 -- June 2017}{Oslo, Norway}

	\cvsection{Languages}

		\cvskill{English}{5}
		\cvskill{Norwegian}{4.5}
		\cvskill{Turkish}{4.5}
		\cvskill{German}{2}

\end{paracol}

\end{document}